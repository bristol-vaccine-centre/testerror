\documentclass[a4paper,10pt]{article}
\usepackage[utf8]{inputenc}

%opening
\title{}
\author{}

\begin{document}

\maketitle

\begin{abstract}

\end{abstract}

\section{}

For any given sensitivity and specificity, one minus the ratio of the 5\% and 10\% estimates of apparent prevalence can be interpreted as an apparent vaccine effectiveness if the true vaccine effectiveness were 50\%. Finally in this artificial scenario the preventable disease burden (\(10\% \times 50\% = 5\%\)) and the difference in the apparent prevalence between the 10\% and 5\% group was calculated for all combinations of sensitivity and specificity. These quantities are related to component sensitivity and specificity graphically, to demonstrate the range of possible outcomes from a fixed scenario under different conditions of test uncertainty.


\end{document}
